%----------------------------------------------------------------------------------------
%	PACKAGES AND OTHER DOCUMENT CONFIGURATIONS
%----------------------------------------------------------------------------------------
\documentclass[margin, centered,lmodern]{res}
\topmargin=-0.5in
\oddsidemargin -.5in
\evensidemargin -.5in
\textwidth=6.5in
\itemsep=0in
\parsep=0in
\newsectionwidth{1in}
\usepackage[pdftex]{graphicx}
\usepackage{graphicx}
\usepackage{enumitem}
\usepackage{wrapfig}
\usepackage{helvet}
\usepackage[colorlinks = true,
            linkcolor = black,
            urlcolor  = black,
            citecolor = black,
            anchorcolor = black]{hyperref}
\setlength{\textwidth}{6.5in} % Text width of the document
\setlength{\textheight}{720pt}
\usepackage{tcolorbox}


\newtcbox{\badge}[1][red]{
  on line, 
  arc=2pt,
  colback=#1!50!black,
  colframe=#1!50!black,
  fontupper=\color{white},
  boxrule=1pt, 
  boxsep=0pt,
  left=6pt,
  right=6pt,
  top=2pt,
  bottom=2pt
}
\begin{document}

%----------------------------------------------------------------------------------------
%	NAME AND ADDRESS SECTION
%----------------------------------------------------------------------------------------\
\setlength{\voffset}{0.3in}
\begin{center}
    \hspace{-\hoffset}
    \huge\bf{\href{https://www.hridaydutta123.github.io}{Hridoy Sankar Dutta}}
    \hfill 
%\smash{\includegraphics[width=2cm]{hridoyd.jpg}}
\end{center}
\vspace{-5mm}
\moveleft\hoffset\vbox{\hrule width 19cm height 0.5pt}
\vspace{-7mm}
\begin{center}
    \hspace{-\hoffset}
	\href{mailto:hridoy.dutta@deakin.edu.au}{hridoy.dutta@deakin.edu.au} / \href{mailto:hridoy.dutta@cl.cam.ac.uk}{hridoy.dutta@cl.cam.ac.uk}  /
	\href{mailto:hridoyd@iiitd.ac.in}{hridoyd@iiitd.ac.in}
\end{center}
\vspace{-7mm}
\begin{resume}

\section{Links}
\textbf{Website:} \url{https://hridaydutta123.github.io} \\
\textbf{Github:} \url{https://github.com/hridaydutta123} \\
\textbf{DBLP:} \url{https://tinyurl.com/hsd-dblp} \\
\textbf{Google Scholar:} \url{https://tinyurl.com/hsd-google-scholar} \\
\textbf{LinkedIn:} \url{https://tinyurl.com/hsd-linkedin}
%----------------------------------------------------------------------------------------
%	WORK EXPERIENCE SECTION
%----------------------------------------------------------------------------------------
\section{Work Experience}
\textbf{Lecturer} \hfill July 2024 - Present\\
\emph{Faculty of Science Engineering and Built Environment/School of Information Technology \\
Deakin University
} \\
\\
\textbf{Research Associate} \hfill February 2021 - June 2024\\
\emph{Department of Computer Science and Technology \& Cambridge Cybercrime Centre \\
University of Cambridge
} \\
\\
\textbf{Assistant Project Engineer} \hfill August 2014 - April 2015 \\
\emph{IIT Guwahati}
Project Title: \textit{Developing Text to Speech System in Assamese Language Phase II} \\
Supervisor: Prof. S.R.M.Prasanna, Professor, Dept. of EEE, Indian Institute of Technology, Guwahati (IITG)



%----------------------------------------------------------------------------------------
%	EDUCATION SECTION
%----------------------------------------------------------------------------------------
\section{Education}
\textbf{PhD in Computer Science and Engineering} \hfill July 2017 - September 2021 \\
\href{http://www.iiitd.ac.in/}{Indraprastha Institute of Information Technology Delhi (IIIT-D)}
\begin{itemize}
 \item Thesis Title: \textit{Detecting and reasoning collusive activities in online media}
 \item Supervisor: \href{https://sites.google.com/site/tanmoychakra88/}{\textbf{Dr. Tanmoy Chakraborty}}, Assistant Professor \& Ramanujan Fellow, Dept. of CSE, Indraprastha Institute of Information Technology Delhi (IIIT-D), India
\end{itemize}
\textbf{M.Tech in Computer Science and Engineering} \hfill 2015 - 2017 \\
\href{http://www.nitdgp.ac.in/}{National Institute of Technology Durgapur, West Bengal, India}
\begin{itemize}
 \item Thesis Title: \textit{An offline crisis mapping system for large-scale natural disasters}
 \item Supervisor: \href{http://www.nitdgp.ac.in/cse/s_nandi/s_nandi.php}{\textbf{Prof. Subrata Nandi}}, Professor, Dept. of CSE, NIT Durgapur and \href{http://www.nitdgp.ac.in/cse/s_saha/s_saha.php}{\textbf{Dr. Sujoy Saha}}, Assistant Professor, Dept. of CSE, NIT Durgapur
 \item CGPA of \textbf{9.15}/10 (Ranked 3rd)
\end{itemize}
\textbf{B.Tech in Computer Science and Engineering} \hfill 2009 - 2013 \\
\href{http://www.gauhati.ac.in/}{Institute of Science and Technology, Gauhati University, Assam, India}
\begin{itemize}
 \item Thesis Title: \textit{Hierarchical Convergecasting with clustering and spanning tree generation}
 \item Supervisor: \href{https://www.iitg.ac.in/sushantak/}{\textbf{Dr. Sushanta Karmakar}}, Associate Professor, Dept. of CSE, Indian Institute of Technology Guwahati (IIT-G)
 \item Mentor: \href{http://cse.iitkgp.ac.in/~sandipc/}{\textbf{Dr. Sandip Chakraborty}}, Associate Professor, Dept. of CSE, Indian Institute of Technology Kharagpur (IIT-KGP)
 \item CGPA of \textbf{9.17}/10
\end{itemize}
%\textbf{High School} - \href{http://www.ssa-school.org/}{Shrimanta Shankar Academy, Guwahati} - \textbf{82.80\%} \hfill 2007 - 2009 \\
%\textbf{Secondary School} - Hindustani Kendriya Vidyalaya - \textbf{81.00\%} \hfill 2000 - 2007
 
%---------------------------------------------------------------------------------------
%	PATENT SECTION
%---------------------------------------------------------------------------------------
\section{Patent}
\begin{itemize}[leftmargin=*]

\item Sujoy Saha,  Prithviraj Pramanik,  Partha Sarathi Paul,  Subrata Nandi,  Kingshuk De,  Bishak Chandra Ghosh,  \textbf{Hridoy Sankar Dutta},  Tamal Mondal,  Krishnendu Hazra,  Indrajit Bhattacharya,  Sandip Chakraborty,  AN END-TO-END SYSTEM FOR OFFLINE LOCALIZED CRISIS MAPPING AND METHOD OF OPERATION THEREOF,  Indian Patent,  Application No: 201931050685,  Filed on 09 December, 2019.

\end{itemize}
%---------------------------------------------------------------------------------------
%	PUBLICATION SECTION
%---------------------------------------------------------------------------------------

\section{Papers under review/Working papers}
\textbf{*} - Equal contribution
\begin{itemize}[leftmargin=*]
\item Luis Adán Saavedra del Toro*,  \textbf{Hridoy Sankar Dutta*}, Alastair Beresford and Alice Hutchings.  Surveying Android and iOS Modded App Market Operators and Developers.
\item Luis Adán Saavedra del Toro*,  \textbf{Hridoy Sankar Dutta*}, Alastair Beresford and Alice Hutchings.  Unveiling the Wild Side of Apps: Exploring the Modded iOS Market Ecosystem.
\item[] \textit{The studies on modded app markets are a part of Google ASPIRE grant.}
\item \textbf{Hridoy Sankar Dutta},  Yanna Papadodimitraki and Alice Hutchings.  A needle in a haystack: A First Look into Marketplaces selling multi-platform Online Media Accounts.
\item Anahit Sargsyan, \textbf{Hridoy Sankar Dutta} and Juergen Pfeffer.  Half-Life of YouTube Videos: A Comparative Analysis of Alternative and Mainstream News Channels in the US.
\end{itemize}


\section{Accepted / Published Papers}
\begin{itemize}[leftmargin=*]
\item[] (\textcolor{blue}{Top conferences and high impact journals are highlighted in blue.}) \\ ** - Equal contribution

\item Luis Adán Saavedra del Toro,  \textbf{Hridoy Sankar Dutta}, Alastair Beresford and Alice Hutchings.  ModZoo: A Large-Scale Study of Modded Android Apps and their Markets,  \textit{APWG Symposium on Electronic Crime Research (eCrime 2024)}, September 24 – 26, 2024, Boston, Massachusetts, USA.
\item Quincy Taylor,  Derek Hansen,  \textbf{Hridoy Sankar Dutta} and Justin Giboney. Analyzing online media platforms for hacktivist group organization and proliferation,  \textit{The 15th Dewald Roode Workshop on Information Systems Security Research (DRW 2023)} 22-23 June 2023, Glasgow, Scotland.

\item Luis Adán Saavedra del Toro,  Alastair Beresford,  \textbf{Hridoy Sankar Dutta} and Alice Hutchings. Analysis of Modded Android Apps and Marketplaces,  \textit{Fourth UK Mobile, Wearable and Ubiquitous Systems Research Symposium (MobiUK)} 4th-5th July 2022, UCL.

\item \textbf{Hridoy Sankar Dutta} and Tanmoy Chakraborty, Blackmarket-driven collusion on online media: a survey.  \textit{\textcolor{blue}{ACM/IMS Transactions on Data Science (TDS)}} 2, no. 4 (2022): 1-37.

\item \textbf{Hridoy Sankar Dutta}, Nirav Diwan, Tanmoy Chakraborty. Weakening the Inner Strength: Spotting Core Collusive Users in YouTube Blackmarket Network,  \textit{\textcolor{blue}{16th International AAAI Conference on Web and Social Media (ICWSM)}.}  vol. 16, pp. 147-158. 2022.

\item \textbf{Hridoy Sankar Dutta}, Mayank Jobanputra, Himani Negi, Tanmoy Chakraborty. Detecting and Analyzing Collusive Entities on YouTube,   \textit{\textcolor{blue}{ACM Transactions on Intelligent Systems and Technology (TIST)}} 12, no. 5 (2021): 1-28.

\item \textbf{Hridoy Sankar Dutta}, Kartik Aggarwal and Tanmoy Chakraborty. DECIFE: Detecting Collusive Users Involved in Blackmarket Following Services on Twitter, \textit{\textcolor{blue}{32nd ACM Conference on Hypertext and Social Media (HyperText)}}, Trinity College Dublin, Ireland,  pp. 91-100. 2021.

\item \textbf{Hridoy Sankar Dutta}, Udit Arora and Tanmoy Chakraborty. ABOME: A Multi-platform Data Repository of Artificially Boosted Online Media Entities, \textit{\textcolor{blue}{15th International AAAI Conference on Web and Social Media (ICWSM)}}, vol. 15, pp. 1000-1008. 2021.

\item \textbf{Hridoy Sankar Dutta} and Tanmoy Chakraborty. Adversarial Collusion on the Web: State-of-the-Art and Future Directions. In Big Data Analytics: 8th International Conference, BDA 2020, Sonepat, India, December 15–18, 2020, Proceedings 8, pp. 230-235. Springer International Publishing, 2020.

\item Udit Arora*,  \textbf{Hridoy Sankar Dutta}*,  Brihi Joshi,  Aditya Chetan and Tanmoy Chakraborty.  Analyzing and detecting collusive users involved in blackmarket retweeting activities. \textit{\textcolor{blue}{ACM Transactions on Intelligent Systems and Technology (TIST)}} 11, no. 3 (2020): 1-24.

\item \textbf{Hridoy Sankar Dutta},  Vishal Raj Dutta,  Aditya Adhikary and Tanmoy Chakraborty. HawkesEye: Detecting fake retweeters using Hawkes process and topic modeling. \textit{\textcolor{blue}{IEEE Transactions on Information Forensics and Security}} 15 (2020): 2667-2678.

\item \textbf{Hridoy Sankar Dutta} and Tanmoy Chakraborty. Blackmarket-driven Collusion among Retweeters – Analysis, Detection and Characterization. \textit{\textcolor{blue}{IEEE Transactions on Information Forensics and Security (TIFS)}} 15 (2019): 1935-1944.

\item Partha Sarathi Paul, Bishakh Chandra Ghosh, \textbf{Hridoy Sankar Dutta}, Kingshuk De, Arka Prava Basu, Prithviraj Pramanik, Sujoy Saha, Sandip Chakraborty, Niloy Ganguly, Subrata Nandi. CRIMP: Here Crisis Mapping Goes Offline. \textit{\textcolor{blue}{Elsevier Journal of Network and Computer Applications (JNCA)}} 146 (2019): 102418.

\item  Aditya Chetan*, Brihi Joshi*, \textbf{Hridoy Sankar Dutta}*, Tanmoy Chakraborty. CoReRank: Ranking to Detect Users Involved in Blackmarket-based Collusive Retweeting Activities. \textit{\textcolor{blue}{12th ACM International Conference on Web Search and Data Mining (WSDM)}}, pp. 330-338. 2019.


\item \textbf{Hridoy Sankar Dutta}, Aditya Chetan, Brihi Joshi, Tanmoy Chakraborty. Retweet Us, We Will Retweet You: Spotting Collusive Retweeters Involved in Blackmarket Services, \textit{IEEE/ACM International Conference on Advances in Social Networks Analysis and Mining} (ASONAM), pp. 242-249. IEEE, 2018.

\item Partha Sarathi Paul, \textbf{Hridoy Sankar Dutta}, Bishakh Chandra Ghosh, Krishnandu Hazra, Sandip Chakraborty, Sujoy Saha, Subrata Nandi. Offline crisis mapping by opportunistic dissemination of crisis data after large-scale disasters, \textit{\textcolor{blue}{2nd ACM SIGSPATIAL International Workshop on the Use of GIS in Emergency Management}}, pp. 1-8. 2016.

\item Sandip Chakraborty, Suchetana Chakraborty, Sushanta Karmakar, \textbf{Hridoy Sankar Dutta}. Hierarchical topology adaptation for distributed convergecast applications, In \textit{29th Annual ACM Symposium on Applied Computing}, pp. 405-407. 2014.\end{itemize}

%----------------------------------------------------------------------------------------
%	FUNDING
%----------------------------------------------------------------------------------------
\section{Funding}
\textbf{Deakin Cyber Collaboration Grants Scheme 2025 (Co-PI)} \hfill 5000 AUD  \\
\emph{Deakin Cyber Research and Innovation Centre} \hfill 2025 - Present

\textbf{OpenAI Researcher Access Program (PI)} \hfill 1000 USD  \\
\emph{OpenAI} \hfill 2025 - Present

\textbf{Deakin University Startup Grant (PI)} \hfill INR 5,00,000  \\
\emph{Deakin University} \hfill 2024 - Present

%----------------------------------------------------------------------------------------
%	TEACHING
%----------------------------------------------------------------------------------------
\section{Teaching}
\textbf{Instructor,  Deakin University} \\
\emph{SIT763: Cyber Security Management} \hfill Nov - Jan 2025 \\
\emph{SIT738: Secure Coding} \hfill July - Oct 2024 \\
\emph{SIT753: Professional Practices in Information Technology} \hfill July - Oct 2024 

\textbf{Course Supervisor and Demonstrator, The University of Cambridge} \\
\emph{Machine Learning and Real-world Data} \hfill Jan - Mar 2022, Jan - Mar 2023 \\
\emph{Pembroke Cambridge Summer Programme} \hfill July - Aug 2022 \\
\emph{Security} \hfill May - June 2021 

\textbf{Teaching Assistant, IIIT Delhi} \\
\emph{Information Retrieval} \hfill Jan - June 2020 \\
\emph{System Management} \hfill August - Nov 2017


\textbf{Teaching Assistant, NIT Durgapur} \\
\emph{Network and Distributed System Laboratory} \hfill Jan - June 2017 \\
\emph{Software Engineering Laboratory} \hfill July - Nov 2016

%----------------------------------------------------------------------------------------
%	THESIS SUPERVISED SECTION
%----------------------------------------------------------------------------------------
\section{Thesis Supervision}

\textbf{Brigham Young University USA} \\
\emph{Role: Honors Thesis Advisor \\Supervisee: Quincy Taylor \\ Project Title: Analyzing Online Media Platforms for Hacktivist Group \\Organization and Proliferation} \hfill January 2023 -  March 2023 \\ \\
\textbf{Ahmedabad University India} \\ 
\emph{Role: Thesis Supervisor \\ Supervisee: Parth Mitesh Shah \\ Project Title: Developing Language-Agnostic thumbnail embeddings\\for YouTube videos} \hfill January 2023 -  April 2023 \\

%----------------------------------------------------------------------------------------
%	OTHER EXPERIENCES SECTION
%----------------------------------------------------------------------------------------
\section{Other Experiences}
\textbf{Co-founder,  \href{https://aaii-axom.github.io/}{Assam AI Initiative (AAII)}} \\
\emph{A non-profit initiative to promote various scopes of AI} \hfill Feb 2021 - Present \\
\\
\textbf{Visiting Researcher, TU Munich} \\
\emph{Supervisor: Prof. Juergen Pfeffer\\ Professor of Computational Social Science \& Big Data, TU Munich} \hfill July 2023 \\
\\
\textbf{Visiting Researcher, Cardiff University} \\
\emph{Supervisor: Prof. Pete Burnap \\ Professor, School of Computer Science and Informatics, Cardiff University} \hfill March 2019,  March 2020 \\
\\
\textbf{Short-Term Scholar, George Mason University} \hfill June 2019 - August 2019\\
\emph{Supervisor: Dr. Hemant Purohit,  Associate Professor, Information Sciences \& Technology Department \\ George Mason University (GMU)}  \\
\\
\textbf{Student Researcher, ForkIT} \\
\emph{Supervisor: {Prof. Subrata Nandi, Dept. of CSE, NIT Durgapur} \& \\ {Dr. Sujoy Saha, Dept. of CSE, NIT Durgapur}} \hfill May 2016 - June 2017 \\
An end-to-end system to generate offline crisis maps \\
\\
\textbf{Summer Research Intern, DISARM Group, NIT Durgapur} \\
\emph{Supervisor: {Prof. Subrata Nandi, Dept. of CSE, NIT Durgapur}} \hfill May 2016 - July 2016 \\
Developed web application admin dashboard for Master Control Station at the time of disaster. Django, a python web framework was used to develop the application.\\
\\
\textbf{Winter Research Intern, DISARM Group, NIT Durgapur} \\
\emph{Supervisor: {Prof. Subrata Nandi, Dept. of CSE, NIT Durgapur}} \hfill December 2015 - January 2016 \\
Developed an opportunistic network for an android application.
\\
\\
\textbf{Project Intern, IIT Guwahati}  \\
\emph{Mentored by {Dr. Sushanta Karmakar, Dept. of CSE, IIT Guwahati}} \hfill January 2013 - June 2013 \\
Worked on the Research Topic "Hierarchical topology adaptation for distributed convergecast applications"


%----------------------------------------------------------------------------------------
%	Selected Projects Section
%----------------------------------------------------------------------------------------
\section{Selected Projects}
All projects are available on git : \url{https://www.github.com/hridaydutta123}
%\setlist[itemize]{
\begin{itemize}[leftmargin=*]
 \item \textbf{\href{https://resource-finder.github.io/}{ResourceFinder}} - A resource map tool created during Covid-19 to locate nearby grocery shops, hospitals and gas stations with their opening time, closing time, phone number etc.

 \item \textbf{\href{https://github.com/hridaydutta123/Assamese2Vec}{Assamese2Vec}} - An open source project for modeling the Assamese language.

 \item \textbf{\href{https://github.com/hridaydutta123/awesome-twitter-tools}{Awesome Twitter Tools}} - A curated list of tools, datasets, research papers and browser extensions related to Twitter. 
 
 \item \textbf{\href{https://github.com/hridaydutta123/the-youtube-scraper}{The YouTube Scraper}} - Download YouTube video description and video comments without using the YouTube API.

 \item \textbf{\href{https://github.com/hridaydutta123/collusive-user-detection-tool}{Collusive User Detection Tool}} - A tool for detection of collusive users in online social networks.

 \item \textbf{\href{https://github.com/ItsForkIT/pdm}{Opportunistic Network for Android Application}} - Implementation of communication medium for challenge network scenario. 
 \item \textbf{\href{http://www.github.com/hridaydutta123/offlinemcs}{Offline Master Control Station}} - Implementation of reliable and effective data transfer from oppurtunistic network and tool for analyzing behaviour of data captured in oppurtunistic networks.
 \item \textbf{\href{http://www.iitg.ernet.in/cseweb/tts/tts/Assamese/}{Designing TTS System in Assamese and Manipuri Languages}} - It involves building systems for synthesizing natural speech in native Assamese and Manipuri tone from a given text and building user friendly GUIs for the common people.

 \item \textbf{\href{https://github.com/hridaydutta123/MapDisarm}{MapDisarm}} - Offline efficient transfer of OSM Map tiles among various devices without internet connectivity for gps based recovery in Oppurtunistic network.
\end{itemize}

%----------------------------------------------------------------------------------------
%	ACHIEVEMENT SECTION
%----------------------------------------------------------------------------------------
\section{Achievements and Awards}
\begin{itemize}[leftmargin=*]
\item Wiseman Prize 2023, Department of Computer Science and Technology, University of Cambridge.
\item DAAD AInet fellowship 2023.
\item Affiliated Postdoctoral Member, Clare College, University of Cambridge.
\item Member of Trinity College Postdoctoral Society,  University of Cambridge.
 \item Awarded ICWSM 2022 Virtual scholarship.
 \item Awarded ICWSM 2021 scholarship for Underrepresented Groups Program.
 \item Selected as a Rising Star in Data Science by the Center for Data and Computing (CDAC), University of Chicago.
 \item Awarded 3rd prize at Ninth IDRBT Doctoral Colloquim IDC 2019.
 \item Awarded Student Travel Grant to attend CODS COMAD 2019.
 \item Grand Challenge Grade A winner at International Symposium on Embedded Devices (ISED) 2016 at IIT Patna.
 \item Came 1st Runners Up in Startup Weekend, Kolkata (5-7th August, 2016).
 \item Media Coverage of our Group in a local daily: \url{https://tinyurl.com/news-coverage-disarm}
 \item Won 3rd Prize in Coding Competition at Techniche 2012 (IIT Guwahati Technical Fest Coding Event).
 \item Attended NASSCOM Product Conclave-2016 as Student Researcher of ForkIT.
 \item Rank of 3 in Master of Technology and Rank of 7 in Bachelor of Technology.
 \item Secured 10/10 SGPA in M.Tech (3rd \& 4th Semester).
 \item Secured 10/10 SGPA in B.Tech (6th, 7th \& 8th Semester).
 \item GATE Qualified (2016) – Gate Score : 488 (Rank: 4370).
 \item GATE Qualified (2015) – Gate Score : 441 (Rank: 6675).
 \item Recipient of GATE Scholarship (2014) – Gate Score : 582 (Rank: 2139).

 \item Passed `A' level examination of National Cadet Crops (NCC).
 % \item Winner of Counter Strike Gaming Competition at Technoshine 2015, Technoshine 2016, Aarohan 2017 and Rectasy 2017 held at NIT Durgapur.
\end{itemize}

\section{Professional Services}
\begin{itemize}[leftmargin=*]
\item Special Interest Group on Information Retrieval (SIGIR 2025 Resource Track) (Reviewer).
\item Pacific-Asia Conference on Knowledge Discovery and Data Mining (PAKDD 2025) (Reviewer).
\item SIAM International Conference on Data Mining (SDM 2025) (Reviewer).
\item IEEE Open Journal of the Computer Society (Reviewer)
\item IEEE Building a Secure & Empowered Cyberspace (BuildSEC 2024) (Poster co-chair)
\item ACM Conference on Hypertext and Social Media (HyperText 2024) (Reviewer)
\item Workshop on Mobility in the Evolving Internet Architecture (MobiCom 2024) (Reviewer)
\item International Conference On Distributed Computing And Networking (ICDCN-SoCIeTY 2023) (Reviewer)
\item ACM International Conference on Web Search and Data Mining (WSDM 2024) (Reviewer)
\item Nature Scientific Reports (Reviewer).
\item Dewald Roode Workshop on Information Systems Security Research (DRW 2023) (Reviewer).
\item Special Interest Group on Information Retrieval (SIGIR 2023 Resource Track) (Reviewer).
\item Transactions on Asian and Low-Resource Language Information Processing (ACM TALLIP) (Reviewer).
\item International Joint Conference on Artificial Intelligence (IJCAI 2023) (Reviewer).
\item ACM Web Conference (WWW 2023) (Reviewer).
\item SIAM International Conference on Data Mining (SIAM 2023) (Reviewer).
\item Springer Machine Learning Journal (Reviewer).
\item Cambridge Cybercrime Conference 2022 (Organizing committee)
\item AAAI Conference on Artificial Intelligence (AAAI 2023) (Reviewer)
\item ACM International Conference on Web Search and Data Mining (WSDM 2023) (Reviewer)
\item International Conference on Neural Information Processing (ICONIP 2022) (Reviewer)
\item ACM International Conference on Information and Knowledge Management (CIKM 2022 Short Paper) (Reviewer)
\item Special Interest Group on Information Retrieval (SIGIR 2022 Resource Track) (Reviewer).
\item Journal of Cybersecurity (Reviewer).
\item IEEE Internet Computing (Reviewer).
\item International Conference on Natural Language Processing (ICON-2021) (Reviewer).
\item SIAM International Conference on Data Mining (SIAM 2022) (Reviewer).
\item Springer Journal on Social Network Analysis and Mining (SNAM)   (Reviewer).
   \item ACM International Conference on Information and Knowledge Management (CIKM 2021 Resource Track) (Reviewer).
    \item International Conference on Natural Language Processing (ICON-2020) (Reviewer)
    \item IEEE International Conference on Advanced Networks and Telecommunications Systems (IEEE ANTS 2020) (Reviewer)
    \item International Joint Conference on Web Intelligence and Intelligent Agent Technology (WI-IAT 2020) (Reviewer)
	\item AAAI Workshop on Combating online Hostile Posts in Regional Languages during Emergency Situation (CONSTRAINT 2021) (Web Chair \& Reviewer)
	\item SIAM International Conference on Data Mining (SIAM 2021) (Reviewer).
	\item ACM International Conference on Information and Knowledge Management (CIKM 2020) (Reviewer).
	\item PAKDD Workshop on Data Science for Fake News (DSFN 2020) (Web Chair \& Reviewer).
	\item Wiley Journal on Expert Systems (Reviewer).
	\item IEEE/WIC/ACM International Conference on Web Intelligence (WI 2019) (External Reviewer).
	\item International Conference on Secure Knowledge Management in Artificial Intelligence Era (SKM 2019) (External Reviewer).
\end{itemize}

%----------------------------------------------------------------------------------------
%	ATTENDED EVENTS SECTION
%----------------------------------------------------------------------------------------
\section{Events \& Workshops Attended}
\begin{itemize}[leftmargin=*]
\item Hidden Populations Workshop, University of Oxford (2022).
\item Security and Human Behaviour (SHB), University of Cambridge (2021).
\item Workshop on AI for Computational Social Systems (ACSS-20), IIIT Delhi (2021).
\item Workshop on Combating Fraud Activities using Data Science (CoFad), IIIT Delhi (2020).
\item Amazon Research Days, Bangalore (2019).
\item Academic Research Summit 2019 on Data Science and AI, IIT Madras, Chennai (2019).
\item ACM India Joint International Conference on Data Science and Management of Data (CoDS-COMAD) (2019).
\item Workshop on AI for Computational Social Systems (ACSS-19), IIIT Delhi (2019).
\item Winter School on Artificial Intelligence,  IIIT Delhi (2019).
\item Amazon Research Days, Bangalore (2018).
\item Young Scientist Conference at India International Science Festival (2018).
\item Workshop on Complex and Social Networks, IIT Kharagpur (2017).
\item NASSCOM Product Conclave, Bangalore (2016).
\end{itemize}

%----------------------------------------------------------------------------------------
%	TECHNICAL SKILLS SECTION
%----------------------------------------------------------------------------------------

\section{Technical \hspace{2mm} Skills}
\textbf{Strongest Areas} - Cybercrime, Data-driven Cybersecurity, Social Network Analysis, Data Mining, Applied Machine Learning, Web Mining, Natural Language Processing \& System Programming. \\
\textbf{Languages} - Python, Java, PHP, Dart, C++, UNIX Shell, HTML, CSS, JS, Ajax, Perl \\
\textbf{Tools} - Git, Android Studio, Adobe Photoshop, JUnit, \LaTeX \\
\textbf{Databases} - Firebase, MySQL, SQLite, MongoDB \\

%----------------------------------------------------------------------------------------
%	PERSONAL DETAILS SECTION
%----------------------------------------------------------------------------------------

\section{Personal Details}
\textbf{Date of Birth:} 12th September, 1991 \\
\textbf{Gender:} Male \\ 
\textbf{Nationality:} Indian

\section{References}
\section{References}
\begin{itemize}[leftmargin=*]
\item[] \textbf{Dr. Tanmoy Chakraborty}, Associate Professor, Dept. of Electrical Engineering,  IIT Delhi
\item[] \textbf{Prof. Alice Hutchings}, Professor, Dept. of CST, University of Cambridge
\item[] \textbf{Prof. Ross Anderson}, Professor, Dept. of CST, University of Cambridge

\end{itemize}

%\item[] \textbf{Dr. Tanmoy Chakraborty} \textless \href{mailto:tanmoy@iiitd.ac.in}{tanmoy@iiitd.ac.in}\textgreater, Associate Professor, Dept. of Electrical Engineering (Computer Technology Group),  IIT Delhi
%\item[] \textbf{Prof. Alice Hutchings} \textless \href{mailto:alice.hutchings@cl.cam.ac.uk}{alice.hutchings@cl.cam.ac.uk}\textgreater, Professor, Dept. of CST, University of Cambridge
%\item[] \textbf{Prof. Ross Anderson} \textless \href{mailto:Ross.Anderson@cl.cam.ac.uk}{Ross.Anderson@cl.cam.ac.uk}\textgreater, Asst. Professor, Dept. of CST, University of Cambridge

\end{resume}
\end{document}