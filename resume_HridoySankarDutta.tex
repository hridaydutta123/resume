%----------------------------------------------------------------------------------------
%	PACKAGES AND OTHER DOCUMENT CONFIGURATIONS
%----------------------------------------------------------------------------------------
\documentclass[margin, centered]{res}
\topmargin=-0.5in
\oddsidemargin -.5in
\evensidemargin -.5in
\textwidth=6.5in
\itemsep=0in
\parsep=0in
\newsectionwidth{1in}
\usepackage[pdftex]{graphicx}
\usepackage{enumitem}
\usepackage{wrapfig}
\usepackage{helvet}
\usepackage[colorlinks = true,
            linkcolor = black,
            urlcolor  = black,
            citecolor = black,
            anchorcolor = black]{hyperref}
\setlength{\textwidth}{6.5in} % Text width of the document
\setlength{\textheight}{720pt}

\begin{document}

%----------------------------------------------------------------------------------------
%	NAME AND ADDRESS SECTION
%----------------------------------------------------------------------------------------\
\setlength{\voffset}{0.3in}
\begin{center}
    \hspace{-\hoffset}
    \huge\bf{\href{https://www.hridaydutta123.github.io}{Hridoy Sankar Dutta}}
    \hfill 
%\smash{\includegraphics[width=2cm]{hridoyd.jpg}}
\end{center}
\vspace{-5mm}
\moveleft\hoffset\vbox{\hrule width 19cm height 0.5pt}
\vspace{-7mm}
\begin{center}
    \hspace{-\hoffset}
	\href{mailto:hridoyd@iiitd.ac.in}{hridoyd@iiitd.ac.in} / \href{mailto:hridaydutta123@gmail.com}{hridaydutta123@gmail.com} \textbullet~ (+91)-8638189794 / (+91)-8135002680 \textbullet~ \#3-E, Pink Enclave, Paltan Bazar, Behind Hotel Starline, Guwahati, Assam, India - 781008
\end{center}
\vspace{-7mm}
\begin{resume}
\section{Links}
\textbf{Website:} \url{https://hridaydutta123.github.io} \\
\textbf{Github:} \url{https://github.com/hridaydutta123/} \\
\textbf{DBLP:} \url{http://dblp.uni-trier.de/pers/hd/d/Dutta:Hridoy_Sankar} \\
\textbf{Google Scholar:} \url{https://scholar.google.co.in/citations?user=4UrQjwUAAAAJ&hl=en} \\
\textbf{LinkedIn:} \url{https://www.linkedin.com/in/hridoy-sankar-dutta-18448a63/?ppe=1} \\

%----------------------------------------------------------------------------------------
%	WORK EXPERIENCE SECTION
%----------------------------------------------------------------------------------------
\section{Job Experience}
\textbf{Research Associate, University of Cambridge} \hfill September 2021 - Present\\
\emph{Cambridge Cybercrime Centre \\
University of Cambridge
} \\
\\
\textbf{Research Assistant, University of Cambridge} \hfill February 2021 - September 2021\\
\emph{Cambridge Cybercrime Centre \\
University of Cambridge
} \\
\\
\textbf{Short-Term Scholar, George Mason University} \hfill June 2019 - August 2019\\
\emph{Mentored by Dr. Hemant Purohit\\
Assistant Professor, Information Sciences \& Technology Department \\ George Mason University}  \\
\\
\textbf{Assistant Project Engineer, IIT Guwahati} \hfill August 2014 - April 2015 \\
Project Title: \textit{Developing Text to Speech System in Assamese Language Phase II} \\
Supervisor: Dr. S.R.M.Prasanna, Professor, Dept. of EEE, Indian Institute of Technology, Guwahati (IIT-G)



%----------------------------------------------------------------------------------------
%	EDUCATION SECTION
%----------------------------------------------------------------------------------------
\section{Education}
\textbf{PhD in Computer Science and Engineering} \hfill August 2017 - September 2021 \\
\href{http://www.iiitd.ac.in/}{Indraprastha Institute of Information Technology Delhi (IIIT-D), India}
\begin{itemize}
 \item Thesis Title: \textit{Detecting and reasoning collusive activities in online media}
 \item Supervisor: \href{https://sites.google.com/site/tanmoychakra88/}{\textbf{Dr. Tanmoy Chakraborty}}, Assistant Professor \& Ramanujan Fellow, Dept. of CSE, Indraprastha Institute of Information Technology Delhi (IIIT-D), India
\end{itemize}
\textbf{M.Tech in Computer Science and Engineering} \hfill 2015 - 2017 \\
\href{http://www.nitdgp.ac.in/}{National Institute of Technology Durgapur, West Bengal, India}
\begin{itemize}
 \item Thesis Title: \textit{An offline crisis mapping system for large-scale natural disasters}
 \item Supervisor: \href{http://www.nitdgp.ac.in/cse/s_nandi/s_nandi.php}{\textbf{Prof. Subrata Nandi}}, Professor, Dept. of CSE, NIT Durgapur and \href{http://www.nitdgp.ac.in/cse/s_saha/s_saha.php}{\textbf{Dr. Sujoy Saha}}, Assistant Professor, Dept. of CSE, NIT Durgapur
 \item CGPA of \textbf{9.15}/10 (Ranked 3rd)
\end{itemize}
\textbf{B.Tech in Computer Science and Engineering} \hfill 2009 - 2013 \\
\href{http://www.gauhati.ac.in/}{Institute of Science and Technology, Gauhati University, Assam, India}
\begin{itemize}
 \item Thesis Title: \textit{Hierarchical Convergecasting with clustering and spanning tree generation}
 \item Supervisor: \href{https://www.iitg.ac.in/sushantak/}{\textbf{Dr. Sushanta Karmakar}}, Associate Professor, Dept. of CSE, Indian Institute of Technology Guwahati (IIT-G)
 \item Mentor: \href{http://cse.iitkgp.ac.in/~sandipc/}{\textbf{Dr. Sandip Chakraborty}}, Assistant Professor, Dept. of CSE, Indian Institute of Technology Kharagpur (IIT-KGP)
 \item CGPA of \textbf{9.17}/10
\end{itemize}
%\textbf{High School} - \href{http://www.ssa-school.org/}{Shrimanta Shankar Academy, Guwahati} - \textbf{82.80\%} \hfill 2007 - 2009 \\
%\textbf{Secondary School} - Hindustani Kendriya Vidyalaya - \textbf{81.00\%} \hfill 2000 - 2007
 
%---------------------------------------------------------------------------------------
%	PATENT SECTION
%---------------------------------------------------------------------------------------
\section{Patent}
\begin{itemize}[leftmargin=*]

\item Sujoy Saha,  Prithviraj Pramanik,  Partha Sarathi Paul,  Subrata Nandi,  Kingshuk De,  Bishak Chandra Ghosh,  \textbf{Hridoy Sankar Dutta},  Tamal Mondal,  Krishnendu Hazra,  Indrajit Bhattacharya,  Sandip Chakraborty,  AN END-TO-END SYSTEM FOR OFFLINE LOCALIZED CRISIS MAPPING AND METHOD OF OPERATION THEREOF,  Indian Patent,  Application No: 201931050685,  Filed on 09 December, 2019.

\end{itemize}
%---------------------------------------------------------------------------------------
%	PUBLICATION SECTION
%---------------------------------------------------------------------------------------
\section{Papers under review}
\begin{itemize}[leftmargin=*]

\item \textbf{Hridoy Sankar Dutta} and Tanmoy Chakraborty. Collusion on the web: A Survey.,  \textit{ACM/IMS Transactions on Data Science}.  (Major Revision)

\item \textbf{Hridoy Sankar Dutta}, Nirav Diwan, Tanmoy Chakraborty. Weakening the Inner Strength: Spotting Core Collusive Users in YouTube Blackmarket Network,  \textit{15th International AAAI Conference on Web and Social Media (ICWSM)}.  (Rebuttal phase)


\end{itemize}

\section{Accepted / Published Papers}
\begin{itemize}[leftmargin=*]
\item[] (\textcolor{blue}{Type-A/A* conferences based on CORE ranking and high impact journals are highlighted in blue.}) \\ ** - Equal contribution

\item \textbf{Hridoy Sankar Dutta}, Mayank Jobanputra, Himani Negi, Tanmoy Chakraborty. Detecting and Analyzing Collusive Entities on YouTube,   \textit{\textcolor{blue}{ACM Transactions on Intelligent Systems and Technology (TIST)}}. (Impact Factor: 3.971)

\item \textbf{Hridoy Sankar Dutta}, Kartik Aggarwal and Tanmoy Chakraborty. DECIFE: Detecting Collusive Users Involved in Blackmarket Following Services on Twitter, \textit{\textcolor{blue}{32nd ACM Conference on Hypertext and Social Media (HyperText)}}, Trinity College Dublin, Ireland,  30 August – 2 September 2021.

\item \textbf{Hridoy Sankar Dutta}, Udit Arora and Tanmoy Chakraborty. ABOME: A Multi-platform Data Repository of Artificially Boosted Online Media Entities, \textit{15th International AAAI Conference on Web and Social Media (ICWSM)}, Atlanta, GA, USA, June 7-10, 2021.

\item Shreyash Arya and \textbf{Hridoy Sankar Dutta}. Revealing the Blackmarket Retweet Game: A Hybrid Approach, \textit{AAAI Workshop on ​Combating On​line Ho​st​ile Posts in ​Regional L​anguages dur​ing Emerge​ncy Si​tuation (CONSTRAINT)}, Feb 8, 2021.

\item \textbf{Hridoy Sankar Dutta} and Tanmoy Chakraborty. Adversarial Collusion on the Web: State-of-the-art and Future Directions,  \textit{The 8th International Conference on Big Data Analytics (BDA)}. (Tutorial)

\item Udit Arora*, \textbf{Hridoy Sankar Dutta*}, Brihi Joshi, Aditya Chetan, Tanmoy Chakraborty. Analyzing and Detecting Collusive Users Involved in Blackmarket Retweeting Activities,  \textit{\textcolor{blue}{ACM Transactions on Intelligent Systems and Technology (TIST)}}. (Impact Factor: 3.971)

\item \textbf{Hridoy Sankar Dutta}, Vishal Raj Dutta, Aditya Adhikary, Tanmoy Chakraborty. HawkesEye: Detecting Fake Retweeters using Hawkes Process and Topic Modeling,  \textit{\textcolor{blue}{IEEE Trans. Information Forensics and Security (TIFS)}}. (Impact Factor: 6.211)

\item \textbf{Hridoy Sankar Dutta} and Tanmoy Chakraborty. Blackmarket-driven Collusion among Retweeters – Analysis, Detection and Characterization. \textit{\textcolor{blue}{IEEE Transactions on Information Forensics and Security (TIFS)}}. (Impact Factor: 6.211)

\item Partha Sarathi Paul, Bishakh Chandra Ghosh, \textbf{Hridoy Sankar Dutta}, Kingshuk De, Arka Prava Basu, Prithviraj Pramanik, Sujoy Saha, Sandip Chakraborty, Niloy Ganguly, Subrata Nandi. CRIMP: Here Crisis Mapping Goes Offline. \textit{\textcolor{blue}{Elsevier Journal of Network and Computer Applications (JNCA)}}. (Impact Factor: 5.273)

\item  Aditya Chetan*, Brihi Joshi*, \textbf{Hridoy Sankar Dutta}*, Tanmoy Chakraborty. CoReRank: Ranking to Detect Users Involved in Blackmarket-based Collusive Retweeting Activities. \textit{\textcolor{blue}{12th ACM International Conference on Web Search and Data Mining (WSDM)}}, Melbourne, Australia, Feb 11-15, 2019. (Acceptance rate: 16\%)


\item \textbf{Hridoy Sankar Dutta}, Aditya Chetan, Brihi Joshi, Tanmoy Chakraborty. Retweet Us, We Will Retweet You: Spotting Collusive Retweeters Involved in Blackmarket Services, \textit{IEEE/ACM International Conference on Advances in Social Networks Analysis and Mining} (ASONAM), Barcelona, Spain, Aug 28-31, 2018. (Acceptance rate: 15\%)

\item Partha Sarathi Paul, \textbf{Hridoy Sankar Dutta}, Bishakh Chandra Ghosh, Krishnandu Hazra, Sandip Chakraborty, Sujoy Saha, Subrata Nandi. Offline crisis mapping by opportunistic dissemination of crisis data after large-scale disasters, \textit{\textcolor{blue}{2nd ACM SIGSPATIAL International Workshop on the Use of GIS in Emergency Management}}, California, United States, Oct 31-Nov 3, 2016.

\item Sandip Chakraborty, Suchetana Chakraborty, Sushanta Karmakar, \textbf{Hridoy Sankar Dutta}. Hierarchical topology adaptation for distributed convergecast applications, In \textit{29th Annual ACM Symposium on Applied Computing}, Gyeongju, Korea, March 24-28, 2014.
\end{itemize}


%----------------------------------------------------------------------------------------
%	TEACHING
%----------------------------------------------------------------------------------------
\section{Teaching}
\textbf{Courses Supervised at The University of Cambridge} \\
\emph{Security} \hfill May - June 2021

\textbf{Teaching Assistant at IIIT Delhi} \\
\emph{Information Retrieval} \hfill Jan - June 2020 \\
\emph{System Management} \hfill August - Nov 2017


\textbf{Teaching Assistant at NIT Durgapur} \\
\emph{Network and Distributed System Laboratory} \hfill Jan - June 2017 \\
\emph{Software Engineering Laboratory} \hfill July - Nov 2016

%----------------------------------------------------------------------------------------
%	OTHER EXPERIENCES SECTION
%----------------------------------------------------------------------------------------
\section{Other Experiences}

\textbf{Visiting Researcher, Cardiff University} \\
\emph{Mentored by Prof. Pete Burnap \\ Professor, School of Computer Science and Informatics, Cardiff University} \hfill March 2019,  March 2020 \\
\\
\textbf{Student Researcher, ForkIT} \\
\emph{Mentored by {Prof. Subrata Nandi, Dept. of CSE, NIT Durgapur} \& \\ {Dr. Sujoy Saha, Dept. of CSE, NIT Durgapur}} \hfill May 2016 - June 2017 \\
An End-to-end system to generate Offline Crisis maps \\
\\
\textbf{Summer Research Intern in DISARM Group, NIT Durgapur} \\
\emph{Mentored by {Prof. Subrata Nandi, Dept. of CSE, NIT Durgapur}} \hfill May 2016 - July 2016 \\
Developed web application Admin Dashboard for Master Control Station at the time of disaster. Django, a python web framework was used to develop the application.\\
\\
\textbf{Winter Research Intern in DISARM Group, NIT Durgapur} \\
\emph{Mentored by {Prof. Subrata Nandi, Dept. of CSE, NIT Durgapur}} \hfill December 2015 - January 2016 \\
Developed an Oppurtunistic Network for Android Application.
\\
\\
\textbf{Project Intern, IIT Guwahati}  \\
\emph{Mentored by {Dr. Sushanta Karmakar, Dept. of CSE, IIT Guwahati}} \hfill January 2013 - June 2013 \\
Worked on the Research Topic "Hierarchical topology adaptation for distributed convergecast applications"


%----------------------------------------------------------------------------------------
%	RELEVANT COURSE SECTION
%----------------------------------------------------------------------------------------
% \section{Certified \hspace{2mm} Courses}
% \begin{itemize}[leftmargin=*]
% \item \textbf{JAVA} Programming Course at InfoConnect Solutions \hfill Aug, 2011 - Feb, 2012 \\
% \end{itemize}
%----------------------------------------------------------------------------------------
%	Selected Projects Section
%----------------------------------------------------------------------------------------
\section{Selected Projects}
All projects are available on git : \url{https://www.github.com/hridaydutta123}
%\setlist[itemize]{
\begin{itemize}[leftmargin=*]
 \item \textbf{\href{https://resource-finder.github.io/}{ResourceFinder}} - A resource map tool created during Covid-19 to locate nearby grocery shops, hospitals and gas stations with their opening time, closing time, phone number etc.

 \item \textbf{\href{https://github.com/hridaydutta123/Assamese2Vec}{Assamese2Vec}} - An open source project for modeling the Assamese language.

 \item \textbf{\href{https://github.com/hridaydutta123/awesome-twitter-tools}{Awesome Twitter Tools}} - A curated list of tools, datasets, research papers and browser extensions related to Twitter.

 \item \textbf{\href{https://github.com/hridaydutta123/the-youtube-scraper}{The YouTube Scraper}} - Download YouTube video description and video comments without using the YouTube API.

 \item \textbf{\href{https://github.com/hridaydutta123/collusive-user-detection-tool}{Collusive User Detection Tool}} - A tool for detection of collusive users in online social networks.

 \item \textbf{\href{https://github.com/ItsForkIT/pdm}{Opportunistic Network for Android Application}} - Implementation of communication medium for challenge network scenario. 
 \item \textbf{\href{http://www.github.com/hridaydutta123/offlinemcs}{Offline Master Control Station}} - Implementation of reliable and effective data transfer from oppurtunistic network and tool for analyzing behaviour of data captured in oppurtunistic networks.
 \item \textbf{\href{http://www.iitg.ernet.in/cseweb/tts/tts/Assamese/}{Designing TTS System in Assamese and Manipuri Languages}} - It involves building systems for synthesizing natural speech in native Assamese and Manipuri tone from a given text and building user friendly GUIs for the common people.

 \item \textbf{\href{https://github.com/hridaydutta123/MapDisarm}{MapDisarm}} - Offline efficient transfer of OSM Map tiles among various devices without internet connectivity for gps based recovery in Oppurtunistic network.
\end{itemize}

%----------------------------------------------------------------------------------------
%	ACHIEVEMENT SECTION
%----------------------------------------------------------------------------------------
\section{Achievements and Awards}
\begin{itemize}[leftmargin=*]
 \item Awarded ICWSM 2021 Scholarship.
 \item Selected in the Rising Stars in Data Science workshop to be organized by the Center for Data and Computing (CDAC), University of Chicago.
 \item Awarded 3rd prize for presenting my work at Ninth IDRBT Doctoral Colloquim IDC 2019.
 \item Awarded Student Travel Grant to attend CODS COMAD 2019.
 \item Grand Challenge Grade \textbf{A} winner at International Symposium on Embedded Devices (ISED) 2016 at IIT Patna.
 \item Came \textbf{1st Runners Up} in Startup Weekend, Kolkata (5-7th August, 2016).
 \item Media Coverage of our Group in a local daily: \url{http://aamarkatha.com/?p=1121#.WMzJSszEghE.whatsapp}
 \item Won \textbf{3rd Prize in Coding Competition} at Techniche 2012 ( IIT Guwahati Technical Fest Coding Event ).
 \item Attended NASSCOM Product Conclave-2016 as Student Researcher of ForkIT
 \item Rank of \textbf{3} in Master Of Technology and Rank of \textbf{7} in Bachelor Of Technology.
 \item Secured 10/10 SGPA in M.Tech (3rd \& 4th Semester)
 \item Secured 10/10 SGPA in B.Tech (6th, 7th \& 8th Semester)
 \item GATE Qualified (2016) – \textbf{Gate Score : 488 (Rank: 4370) }
 \item GATE Qualified (2015) – \textbf{Gate Score : 441 (Rank: 6675) }
 \item Recipient of GATE Scholarship (2014) – \textbf{Gate Score : 582 (Rank: 2139) }

 \item Passed `A' level examination of National Cadet Crops (NCC).
 % \item Winner of Counter Strike Gaming Competition at Technoshine 2015, Technoshine 2016, Aarohan 2017 and Rectasy 2017 held at NIT Durgapur.
\end{itemize}

\section{Professional Services}
\begin{itemize}[leftmargin=*]
\item IEEE Internet Computing (Reviewer).
\item International Conference on Natural Language Processing (ICON-2021) (Reviewer).
\item SDM 2022: SIAM International Conference on Data Mining (SIAM 2022) (Reviewer).
\item Springer Journal on Social Network Analysis and Mining (SNAM)   (Reviewer).
   \item ACM International Conference on Information and Knowledge Management (CIKM 2021 Resource Track) (Reviewer).
    \item International Conference on Natural Language Processing (ICON-2020) (Reviewer)
    \item IEEE International Conference on Advanced Networks and Telecommunications Systems (IEEE ANTS 2020) (Reviewer)
    \item International Joint Conference on Web Intelligence and Intelligent Agent Technology (WI-IAT 2020) (Reviewer)
	\item AAAI Workshop on ​ Combating ​ On​line Ho​st​ile Posts in ​ Regional L​anguages dur​ing Emerge​ncy Si​tuation (CONSTRAINT 2021) (Web Chair & Reviewer)
	\item SDM 2021: SIAM International Conference on Data Mining (SIAM 2021) (Reviewer).
	\item ACM International Conference on Information and Knowledge Management (CIKM 2020) (Reviewer).
	\item PAKDD Workshop on Data Science for Fake News (DSFN 2020) (Web Chair and Reviewer).
	\item Wiley Journal on Expert Systems (Reviewer).
	\item IEEE/WIC/ACM International Conference on Web Intelligence (WI 2019) (External Reviewer).
	\item International Conference on Secure Knowledge Management in Artificial Intelligence Era (SKM 2019) (External Reviewer).
\end{itemize}

%----------------------------------------------------------------------------------------
%	ATTENDED EVENTS SECTION
%----------------------------------------------------------------------------------------
\section{Events \& Workshops Attended}
\begin{itemize}[leftmargin=*]
\item Security and Human Behaviour (SHB), University of Cambridge (2021)
\item Amazon Research Days, Bangalore (2019)
\item Academic Research Summit 2019 on Data Science and AI, IIT Madras, Chennai (2019)
\item ACM India Joint International Conference on Data Science and Management of Data (CoDS-COMAD 2019)
\item Winter School on Artificial Intelligence, IIIT Delhi (2019)
\item Amazon Research Days, Bangalore (2018)
\item Young Scientist Conference at India International Science Festival (2018)
\item Workshop on Complex and Social Networks, IIT Kharagpur (2017)
\item NASSCOM Product Conclave, Bangalore (2016)
\end{itemize}

%----------------------------------------------------------------------------------------
%	TECHNICAL SKILLS SECTION
%----------------------------------------------------------------------------------------

\section{Technical \hspace{2mm} Skills}
\textbf{Strongest Areas} - Social Network Analysis, Data Mining, Applied Machine Learning, Web Mining, Natural Language Processing \& System Programming, Web Technology.  \\
\textbf{Languages} - Python, Java, PHP, Dart, C++, UNIX Shell, HTML, CSS, JS, Ajax, Perl \\
\textbf{Frameworks} - Twitter API Clients, NLTK, Sklearn, Gensim, Django, Flask, Flutter \\
\textbf{Tools} - Git, Android Studio, Adobe Photoshop, JUnit, \LaTeX \\
\textbf{Databases} - Firebase, MySQL, SQLite, MongoDB \\

%----------------------------------------------------------------------------------------
%	PERSONAL DETAILS SECTION
%----------------------------------------------------------------------------------------

\section{Personal Details}
\textbf{Date of Birth:} 12th September, 1991 \\
\textbf{Gender:} Male \\ 
\textbf{Nationality:} Indian

\section{References}
\begin{itemize}[leftmargin=*]
\item[] \textbf{Dr. Tanmoy Chakraborty} \textless \href{mailto:tanmoy@iiitd.ac.in}{tanmoy@iiitd.ac.in}\textgreater, Asst. Professor, Dept. of CSE, IIIT Delhi
\item[] \textbf{Dr. Sandip Chakraborty} \textless \href{mailto:sandipc@cse.iitkgp.ac.in}{sandipc@cse.iitkgp.ac.in}\textgreater, Asst. Professor, Dept. of CSE, IIT Kharagpur
\item[] \textbf{Prof. Subrata Nandi} \textless \href{mailto:subrata.nandi@cse.nitdgp.ac.in}{subrata.nandi@cse.nitdgp.ac.in}\textgreater, Professor, Dept. of CSE, NIT Durgapur
\item[] \textbf{Dr. Bibhash Sen} \textless \href{mailto:bibhash.sen@cse.nitdgp.ac.in}{bibhash.sen@cse.nitdgp.ac.in}\textgreater, Assoc. Professor, Dept. of CSE, NIT Durgapur

\end{itemize}


\end{resume}
\end{document}